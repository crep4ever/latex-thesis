\newcommand{\fogrimmi}{FoGrImMi\xspace}
\newcommand{\siecle}[1]{\textsc{#1}\ieme~siècle}
\newcommand{\orbit}[1]{\langle{#1}\rangle}
\newcommand{\card}[1]{|{#1}|}
\newcommand{\abs}[1]{|{#1}|}
\newcommand{\nb}[1]{{\#}{#1}}

\newcommand{\surfels}[1]{\ensuremath{\mbox{\em surfels}(#1)}}
\newcommand{\lignels}[1]{\ensuremath{\mbox{\em lignels}(#1)}}
\newcommand{\pointels}[1]{\ensuremath{\mbox{\em pointels}(#1)}}
\newcommand{\pixels}[1]{\ensuremath{\mbox{\em pixels}(#1)}}
\newcommand{\swap}[1]{\ensuremath{\mbox{\em bascule}(#1)}}
\newcommand{\region}[1]{\ensuremath{\mbox{\em r\'egion}(#1)}}
\newcommand{\face}[1]{\ensuremath{\mbox{\em face}(#1)}}
\newcommand{\triplet}[1]{\ensuremath{\mbox{\em triplet}(#1)}}
\newcommand{\surfaces}[1]{\ensuremath{\mbox{\em surfaces}(#1)}}
\renewcommand{\sf}[1]{\ensuremath{\mbox{\em sf}(#1)}}
\renewcommand{\deg}[1]{\ensuremath{\mbox{\em d}(#1)}}
\newcommand{\surfel}[1]{\ensuremath{\mbox{\em surfel}(#1)}}
\newcommand{\lignel}[1]{\ensuremath{\mbox{\em lignel}(#1)}}
\newcommand{\pointel}[1]{\ensuremath{\mbox{\em pointel}(#1)}}

\newcommand{\brins}[1]{\ensuremath{\mbox{\em brins}(#1)}}
\newcommand{\faces}[1]{\ensuremath{\mbox{\em faces}(#1)}}
\newcommand{\regions}[1]{\ensuremath{\mbox{\em regions}(#1)}}

%\newcommand{\father}[1]{\ensuremath{\mbox{\em pere}(#1)}}
%\newcommand{\rep}[1]{\ensuremath{\mbox{\em rep}(#1)}}
%\newcommand{\sons}[1]{\ensuremath{\mbox{\em fils}(#1)}}
\newcommand{\cc}[1]{\ensuremath{\mbox{\em CC}(#1)}}

\newcommand{\voxels}[1]{\ensuremath{\mbox{\em voxels}(#1)}}
\newcommand{\premierVoxel}[1]{\ensuremath{\mbox{\em premierVoxel}(#1)}}

\newcommand{\voxelIn}[1]{\ensuremath{\mbox{\em voxel}_1(#1)}}
\newcommand{\voxelOut}[1]{\ensuremath{\mbox{\em voxel}_2(#1)}}

\newcommand{\union}[1]{\ensuremath{\mbox{\em union}(#1)}}
\newcommand{\find}[1]{\ensuremath{\mbox{\em trouver}(#1)}}
\newcommand{\trouve}[1]{\ensuremath{\mbox{\em trouver}(#1)}}

\newcommand{\inner}[1]{\ensuremath{\mbox{\em inner}(#1)}}
%\newcommand{\outer}[1]{\ensuremath{outer(#1)}}

\newcommand{\mst}[2]{\ensuremath{\mbox{\em MST}(#1,#2)}}
\newcommand{\cint}[1]{\ensuremath{\mbox{\em Int}(#1)}}
\newcommand{\cext}[2]{\ensuremath{\mbox{\em Ext}(#1,#2)}}
\newcommand{\cextf}[1]{\ensuremath{\mbox{\em Ext}(#1)}}
\newcommand{\mcint}[2]{\ensuremath{\mbox{\em MInt}(#1,#2)}}

\newcommand{\oracle}[2]{\ensuremath{\mbox{\em Oracle}(#1,#2)}}

\newcommand{\ie}{c'est-à-dire,\xspace}
\newcommand{\eg}{par exemple}
\newcommand{\etc}{etc.\xspace}
\newcommand{\apriori}{\emph{a priori}}
\newcommand{\adhoc}{\emph{ad-hoc}}
\newcommand{\anglais}[1]{(\emph{#1} en anglais)}

%\newcommand{\degree}{\ensuremath{\mbox{\em degr\'e}}}

\newtheorem{theorem}{Th\'eor\`eme}
\newtheorem{corolary}{Corrolaire}
\newtheorem{lemma}{Lemme}
\newtheorem{definition}{D\'efinition}
\newtheorem{proposition}{Proposition}
%\newtheorem{proof}{Démonstration}

\newcommand{\todo}[1]{\par{\bf{TODO: #1}}\par}

\newcommand{\dart}{b} 
\newcommand{\Darts}{\mathcal{B}}
\newcommand{\Regions}{\mathcal{R}} 
\newcommand{\Pixels}{\mathcal{P}} 
\newcommand{\betti}{\mathfrak{b}}
\newcommand{\vertex}{s}
\newcommand{\edge}{a}
\newcommand{\vface}{f}
\newcommand{\voxel}{v}
\newcommand{\vsurface}{\mathfrak{s}}

\newcommand{\geomap}{GeoMap}

\newcommand{\Z}{\mathbb{Z}}
\newcommand{\R}{\mathbb{R}}
\newcommand{\N}{\mathbb{N}}

%pyramides
\newcommand{\LivingDarts}{\ensuremath{\mbox{\em $\mathcal{BS}$}}\xspace}
\newcommand{\NonLivingDarts}{\ensuremath{\mbox{\em $\mathcal{BN}$}}\xspace}
\newcommand{\Vertices}{\ensuremath{\mbox{\em $\mathcal{S}$}}\xspace}
\newcommand{\LivingVertices}{\ensuremath{\mbox{\em $\mathcal{SS}$}}\xspace}
\newcommand{\NonLivingVertices}{\ensuremath{\mbox{\em $\mathcal{SN}$}}\xspace}
\newcommand{\Edges}{\ensuremath{\mbox{\em $\mathcal{A}$}}\xspace}
\newcommand{\Pointels}{\ensuremath{\mbox{\em $\mathcal{P}$}}\xspace}
\newcommand{\Lignels}{\ensuremath{\mbox{\em $\mathcal{L}$}}\xspace}
\newcommand{\LivingEdges}{\ensuremath{\mbox{\em $\mathcal{AS}$}}\xspace}
\newcommand{\NonLivingEdges}{\ensuremath{\mbox{\em $\mathcal{AN}$}}\xspace}
\newcommand{\OrientedGraph}{\ensuremath{\mbox{\em $\mathcal{GO}$}}\xspace}
\newcommand{\NonOrientedGraph}{\ensuremath{\mbox{\em $\mathcal{GNO}$}}\xspace}
\newcommand{\cells}[1]{\ensuremath{\mbox{\em $#1$-cellules}}}
\newcommand{\cell}[1]{\ensuremath{\mbox{\em $#1$-cellule}}}
\newcommand{\cycle}{\ensuremath{\mbox{\em $\pi$}}}

\newcommand{\ReceptiveField}{\ensuremath{\mbox{\em $\mathcal{CR}$}}}
\newcommand{\ConnectingWalk}{\ensuremath{\mbox{\em $\mathcal{CW}$}}}
\newcommand{\ConnectingSequence}{\ensuremath{\mbox{\em $\mathcal{SC}$}}}
\newcommand{\ProjectionStrip}{\ensuremath{\mbox{\em $\mathcal{BP}$}}\xspace}
\newcommand{\Kernel}{\ensuremath{\mbox{\em $\mathcal{K}$}}}
\newcommand{\ssi}{\ensuremath{\mbox{\em ~$\Leftrightarrow$~}}}
\newcommand{\resp}{\emph{resp.}\xspace}
\newcommand{\id}{\ensuremath{\mbox{\em $Id.$}}}
\newcommand{\updown}{\emph{prédécesseur/successeur\xspace}}
\newcommand{\etou}{et/ou\xspace}
\newcommand{\command}[1]{{\mdseries\ttfamily\upshape#1}}
\newcommand{\ligne}[1]{{\mdseries\ttfamily\upshape{ligne\,#1}}}
\renewcommand{\unit}[1]{\,{\mdseries\ttfamily\upshape#1}}
\newcommand{\xpos}{\ensuremath{\mbox{\em x}}}
\newcommand{\xneg}{\ensuremath{\mbox{\em -x}}}
\newcommand{\ypos}{\ensuremath{\mbox{\em y}}}
\newcommand{\yneg}{\ensuremath{\mbox{\em -y}}}
\newcommand{\firstpixel}{\emph{premier pixel}\xspace}


\makeatletter

\newcommand{\ReductionWindow}[1]{
  \@ifmtarg{#1}{%
    \ensuremath{\mbox{\em $\mathcal{FR}$}}
  }{%
    \ensuremath{\mbox{\em $\mathcal{FR}_{#1}$}}
  }%
}

\newcommand{\father}[1]{%
  \@ifmtarg{#1}{%
    \emph{p\`ere}%
  }{%
    \ensuremath{\mbox{\em p\`ere}(#1)}%
  }%
}
\newcommand{\sons}[1]{%
  \@ifmtarg{#1}{%
    \emph{fils}%
  }{%
    \ensuremath{\mbox{\em fils}(#1)}%
  }%
}
\newcommand{\rep}[1]{%
  \@ifmtarg{#1}{%
    \emph{rep}%
  }{%
    \ensuremath{\mbox{\em rep}(#1)}%
  }%
}
\newcommand{\pred}[1]{%
  \@ifmtarg{#1}{%
    \emph{pr\'ed\'ecesseur}%
  }{%
    \ensuremath{\mbox{\em pr\'ed\'ecesseur}(#1)}%
  }%
}
\renewcommand{\succ}[1]{%
  \@ifmtarg{#1}{%
    \emph{successeur}%
  }{%
    \ensuremath{\mbox{\em successeur}(#1)}%
  }%
}
\newcommand{\parametres}[1]{%
  \@ifmtarg{#1}{%
    \emph{paramètres}%
  }{%
    \ensuremath{\mbox{\em param\`etres}(#1)}%
  }%
}%
\makeatother

\newcommand{\Tiles}{\ensuremath{\mbox{\em $\mathcal{T}$}}\xspace}
\newcommand{\Images}{\ensuremath{\mbox{\em $\mathcal{I}$}}\xspace}
\newcommand{\Graphs}{\ensuremath{\mbox{\em $\mathcal{G}$}}\xspace}
\newcommand{\Boule}{\ensuremath{\mbox{\em B}}\xspace}


% Internal commands to handle references
% To reference a figure
\newcommand{\Reffig}[1]{Figure~\ref{#1}} % in the begining of a phrase
\newcommand{\reffig}[1]{Figure~\ref{#1}} % in the middle of a phrase

% To reference a part
\newcommand{\Refpart}[1]{Partie~\ref{#1}} % in the begining of a phrase
\newcommand{\refpart}[1]{Partie~\ref{#1}} % in the middle of a phrase

% To reference a chapter
\newcommand{\Refchap}[1]{Chapitre~\ref{#1}} % in the begining of a phrase
\newcommand{\refchap}[1]{Chapitre~\ref{#1}} % in the middle of a phrase

% To reference a section
\newcommand{\Refsec}[1]{Section~\ref{#1}} % in the begining of a phrase
\newcommand{\refsec}[1]{Section~\ref{#1}} % in the middle of a phrase

% To reference a table
\newcommand{\Reftab}[1]{Tableau~\ref{#1}} % in the begining of a phrase
\newcommand{\reftab}[1]{Tableau~\ref{#1}} % in the middle of a phrase

% To reference an algorithm
\newcommand{\Refalgo}[1]{Algorithme~\ref{#1}} % in the begining of a phrase
\newcommand{\refalgo}[1]{Algorithme~\ref{#1}} % in the middle of a phrase

% To reference a definition
\newcommand{\Refdef}[1]{Définition~\ref{#1}} % in the begining of a phrase
\newcommand{\refdef}[1]{Définition~\ref{#1}} % in the middle of a phrase

% To reference a proposition
\newcommand{\Refprop}[1]{Proposition~\ref{#1}} % in the begining of a phrase
\newcommand{\refprop}[1]{Proposition~\ref{#1}} % in the middle of a phrase

% To reference a corolary
\newcommand{\Refcor}[1]{Corollaire~\ref{#1}} % in the begining of a phrase
\newcommand{\refcor}[1]{Corollaire~\ref{#1}} % in the middle of a phrase

% To reference a lemma
\newcommand{\Reflemma}[1]{Lemme~\ref{#1}} % in the begining of a phrase
\newcommand{\reflemma}[1]{Lemme~\ref{#1}} % in the middle of a phrase
\newcommand{\condition}[1]{condition~(#1)}

\newcommand\green{$+$\xspace}
\newcommand\red{$-$\xspace}
\newcommand\orange{$+/-$\xspace}
\newcommand\bottomup{$\Uparrow$\xspace}
\newcommand\topdown{$\Downarrow$\xspace}

\newcommand\adj[2]{\ensuremath{#1~\mathrm{adj}~#2}}

\def\restriction#1#2{\mathchoice
              {\setbox1\hbox{${\displaystyle #1}_{\scriptstyle #2}$}
              \restrictionaux{#1}{#2}}
              {\setbox1\hbox{${\textstyle #1}_{\scriptstyle #2}$}
              \restrictionaux{#1}{#2}}
              {\setbox1\hbox{${\scriptstyle #1}_{\scriptscriptstyle #2}$}
              \restrictionaux{#1}{#2}}
              {\setbox1\hbox{${\scriptscriptstyle #1}_{\scriptscriptstyle #2}$}
              \restrictionaux{#1}{#2}}}
\def\restrictionaux#1#2{{#1\,\smash{\vrule height .8\ht1 depth .85\dp1}}_{\,#2}}


\newcommand{\sizeof}[1]{\ensuremath{\mbox{\em taille(#1)}}}
\newcommand{\moy}{\ensuremath{\mbox{\em moy}}}
\newcommand{\var}{\ensuremath{\mbox{\em var}}}
\newcommand{\threshold}{\ensuremath{\mbox{\em seuil}}}
\newcommand{\etal}[1]{#1~\emph{et~al.}}
\newcommand{\cf}[1]{(voir~\refsec{#1})}

\newcommand{\ram}[1]{{\mdseries\ttfamily\upshape ram}(#1)}
\newcommand{\topo}{topologie}
\newcommand{\tree}{arbre}
\newcommand{\geom}{géométrie}
\newcommand{\tq}{~|~}
\newcommand{\col}[1]{colonne~#1}
\newcommand{\cols}[1]{colonnes~#1}

\let\textORIG\text
\renewcommand{\text}[1]{%
  \textORIG{~#1~}%
}
